\documentclass{beamer}
\usetheme{default}
\usecolortheme{default}

\title{BT: Identifying good traces in very large Datasets}
\subtitle{Clustering of Singular Values}
\author{}
\date{\today}

\begin{document}

\frame{\titlepage}

\begin{frame}
\frametitle{The H-Matrix}
\begin{itemize}
    \item 4x4 channel matrix $H \in \mathbb{C}^{4 \times 4}$
    \item Rows receivers, columns transmitters
    \item $H_{i,j}=$
    Amplitude and phase received by receiver i transmitted by transmitter j

    \item Current approach: \[
\kappa(H) = \frac{\sigma_{\max}}{\sigma_{\min}}
\]
where $\sigma_{\max},\sigma_{\min}$ are max and min Singular Values
\end{itemize}

\end{frame}

\begin{frame}
\frametitle{Data}
\centering
\includegraphics[width=0.8\textwidth]{images/city.png}
\end{frame}

\begin{frame}
\frametitle{Conditioning Numbers}
\centering
\includegraphics[width=0.8\textwidth]{images/con_num.png}
\end{frame}

\begin{frame}
\frametitle{Singular Values}
\centering
\includegraphics[width=0.8\textwidth]{images/SVs.png}
\end{frame}

\begin{frame}
\frametitle{Capacity}
\begin{itemize}
    \item<1-> Channel Capacity $C$ indicates maximum data rate
    \item<1-> Depends on H-Matrix and $\rho$ (Signal to Noise Ratio)
    \item<1-> Formula: \[
C = \log_2 \left( \det \left( I + \frac{\rho}{N_t} H H^{\dagger} \right) \right)
\]
    where $N_t$ is number of transmit antennas
    \item<1-> Higher capacity = better channel quality
    \item<2-> Can be expressed using singular values: \[C = \sum_{i=1}^{4} \log_2 \left( 1 + \frac{\rho}{N_t} \sigma_i^2 \right)\]
\end{itemize}
\end{frame}

\begin{frame}
\frametitle{Capacity over Time}
\centering

\includegraphics[width=0.5\textwidth]{images/city.png}
\includegraphics[width=0.5\textwidth]{images/cap.png}
\end{frame}

\begin{frame}
\frametitle{Proposal}

\begin{itemize}
    \item<1-> Look at Singular Values
    \item<2-> 4D instead of 1D
    \item<3-> Analyze the 4D problem
\end{itemize}
\end{frame}

\begin{frame}
\frametitle{Analysis}
\begin{itemize}
        \item<1-> \textbf{Approach 1: Singular Value Clustering}
        \begin{itemize}
            \item Cluster the singular values over time
            \item Analyse 1D time series of cluster number
            \item Change of cluster could indicate a changepoint
        \end{itemize}
        \item<2-> \textbf{Approach 2: Autocorrelation Function}
        \begin{itemize}
            \item Compute the Autocorrelation function
            \item Expect a sharp decline over time for a stationary series
            \item Assume markovian structure in time series
        \end{itemize}
    \end{itemize}

\end{frame}






\end{document}